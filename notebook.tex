\documentclass[10pt]{article}
\usepackage[left=0.4in,right=0.4in,top=0.7in,bottom=0.4in]{geometry}
\usepackage{hyperref}
\usepackage{fancyhdr}
\usepackage{listings}
\usepackage{xcolor}
\usepackage{tocloft}
\usepackage{pdflscape}
\usepackage{multicol}
\usepackage{parskip}
\usepackage[spanish]{babel}
\usepackage[utf8]{inputenc}
\usepackage{amsmath}

\renewcommand*{\ttdefault}{pcr}
\renewcommand\cftsecfont{\fontsize{8}{9}\bfseries}
\renewcommand\cftsecpagefont{\fontsize{8}{9}\mdseries}
\renewcommand\cftsubsecfont{\fontsize{5}{6}\mdseries}
\renewcommand\cftsubsecpagefont{\fontsize{5}{6}\mdseries}
\renewcommand\cftsecafterpnum{\vspace{-1ex}}
\renewcommand\cftsubsecafterpnum{\vspace{-1ex}}

\lstdefinestyle{shared}{
    belowcaptionskip=1\baselineskip,
    breaklines=true,
    xleftmargin=\parindent,
    showstringspaces=false,
    basicstyle=\fontsize{8}{6}\ttfamily,
}
\lstdefinestyle{cpp}{
	style=shared,
    language=C++,
    keywordstyle=\bfseries\color{green!40!black},
    commentstyle=\itshape\color{red!80!black},
    identifierstyle=\color{blue},
    stringstyle=\color{purple!40!black},
}
\lstdefinestyle{java}{
    style=shared,
    language=Java,
    keywordstyle=\bfseries\color{green!40!black},
    commentstyle=\itshape\color{purple!40!black},
    identifierstyle=\color{blue},
    stringstyle=\color{orange},
}
\lstdefinestyle{py}{
    style=shared,
    language=Python,
    keywordstyle=\bfseries\color{green!40!black},
    commentstyle=\itshape\color{purple!40!black},
    identifierstyle=\color{blue},
    stringstyle=\color{orange},
}
\lstdefinestyle{txt}{
    style=shared,
}
\lstset{escapechar=@}

\pagestyle{fancy}
\fancyhead[L]{UCLV Team Reference}
\fancyhead[R]{\thepage}
\fancyfoot[C]{}

\fancypagestyle{plain}
{
\fancyhead[L]{UCLV Team Reference}
\fancyhead[R]{\thepage}
\fancyfoot[C]{}
}

\title{\vspace{-4ex}\Large{KFP Team Reference. Universidad Central de las Villas, UCLV}}
\author{}
\date{}

\begin{document}
\begin{landscape}

\maketitle
\vspace{-13ex}

\begin{multicols}{2}

\tableofcontents
\pagestyle{fancy}

\input contents.tex

\end{multicols}

\pagebreak

\section{Fórmulas}

\subsection{Teorema de Pick}

$A = i + \frac{b}{2} - 1$

Notas:
\begin{itemize}
	\item $A$ es el área de un polígono simple
	\item $i$ es el número de puntos interiores del polígono
	\item $b$ es el número de puntos en el borde del polígono
\end{itemize}

\subsection{Número de puntos enteros entre dos puntos}

$\gcd (\lvert x1 - x2 \rvert, \lvert y1 - y2 \rvert) + 1$

Notas:
\begin{itemize}
	\item las coordenadas de los puntos son $(x1, y1), (x2, y2)$
	\item esto cuenta ambos puntos extremos
\end{itemize}

\subsection{Fórmula de Euler para grafos planares conexos}

$V - E + F = 2$

Notas:
\begin{itemize}
    \item $V$ es el número de vértices del grafo
    \item $E$ es el número de aristas del grafo
    \item $F$ es el número de caras del grafo
\end{itemize}

\subsection{Intersección de rectas, segmentos o rayos (forma paramétrica)}

Rectas que pasan por puntos $a, b$ y $c, d$ respectivamente:

$\displaystyle \vec{v_1} = b - a$; $\displaystyle \vec{v_2} = d - c$; $\displaystyle p_1 = a$; $\displaystyle p_2 = c$

$\displaystyle p_1 + \alpha \cdot \vec{v_1} = p_2 + \beta \cdot \vec{v_2}$

$\displaystyle \alpha = \frac{ \left( p_2 - p_1 \right) \times \vec{v_2} } { \vec{v_1} \times \vec{v_2} }$

$\displaystyle \beta = \frac{ \left( p_1 - p_2 \right) \times \vec{v_1} }{ \vec{v_2} \times \vec{v_1} }$

\end{landscape}
\end{document}
